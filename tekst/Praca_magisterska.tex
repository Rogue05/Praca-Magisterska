%Przykładowy plik ułatwiający złożenie projektu dyplomowego inżynierskiego.
%UWAGA: Generowany napis na stronie tytułowej o treści PROJEKT DYPLOMOWY INŻYNIERSKI został zaproponowany przeze mnie i nie jest, póki co, potwierdzony przez władze wydziału. Przed ostatecznym oddaniem tak złożonej pracy należy upewnić się jaka powinna być treść tego napisu. W momencie gdy uzyskam informację na temat treści tego napisu, dokonam niezbędnych zmian w źródłach.

%\documentclass[eng,printmode]{mgr}
\documentclass{mgr}

%opcje klasy dokumentu mgr.cls zostały opisane w dołączonej instrukcji

\usepackage{longtable}

%\input{arduinoLanguage.tex}
%\usepackage[hyphens]{url}
%\usepackage{minted}
\usepackage[newfloat]{minted}
\usepackage{caption}
\usepackage{xcolor} 
\newenvironment{code}{\captionsetup{type=listing}}{}
%\SetupFloatingEnvironment{listing}{name=Source Code}

\definecolor{codeGray}{RGB}{240,240,240} 
\definecolor{codeBlack}{RGB}{0,0,0} 
\definecolor{codeRed}{RGB}{221,0,0} 
\definecolor{codeBlue}{rgb}{0,0,187} 
\definecolor{codeYellow}{RGB}{255,128,0} 
\definecolor{codeGreen}{RGB}{0,119,0}
\usepackage{gensymb}
%poniżej deklaracje użycia pakietów, usunąć to co jest niepotrzebne
%\usepackage{polski} %przydatne podczas składania dokumentów w j. polskim
\usepackage[polish]{babel}%alternatywnie do pakietu polski, wybrać jeden z nich
%\usepackage[cp1250]{inputenc} %kodowanie znaków, zależne od systemu
\usepackage[utf8]{inputenc}
\usepackage[T1]{fontenc} %poprawne składanie polskich czcionek

%pakiety do grafiki
\usepackage{graphicx}
\usepackage{subfigure}
\usepackage{psfrag}

%pakiety dodające dużo dodatkowych poleceń matematycznych
\usepackage{amsmath}
\usepackage{amsfonts}

%pakiety wspomagające i poprawiające składanie tabel
\usepackage{supertabular}
\usepackage{array}
\usepackage{tabularx}
\usepackage{hhline}

%pakiet wypisujący na marginesie etykiety równań i rysunków zdefiniowanych przez \label{}, chcąc wygenerować finalną wersję dokumentu wystarczy usunąć poniższą linię
%\usepackage{showlabels}
%definicje własnych poleceń
\newcommand{\R}{I\!\!R} %symbol liczb rzeczywistych, działa tylko w trybie matematycznym
\newtheorem{theorem}{Twierdzenie}[section] %nowe otoczenie do składania twierdzeń



% moje libki start
\usepackage{tikz-cd}
\usepackage{enumitem}

\usepackage[ruled,vlined]{algorithm2e}
%\usepackage[ruled]{algorithm2e}
\renewcommand{\algorithmcfname}{Algorytm}

\usepackage{float}

%\usepackage[polish]{babel}
%\babelprovide[transforms = oneletter.nobreak]{polish}


%\clubpenalty=10000


%\usepackage[all]{nowidow}

\let\oldref\ref
\renewcommand{\ref}[1]{(\oldref{#1})}

%\usepackage{titlesec}
%\titlespacing*{\section}
%{0pt}{5.5ex plus 1ex minus .2ex}{4.3ex plus .2ex}

\setlength{\parskip}{0cm}
\setlength{\parindent}{1em}
\usepackage[compact]{titlesec}
\titlespacing{\section}{0pt}{4ex}{1ex}
\titlespacing{\subsection}{0pt}{2ex}{0ex}
\titlespacing{\subsubsection}{0pt}{1ex}{0ex}

\usepackage[nodisplayskipstretch]{setspace}
%\setlength{\abovedisplayskip}{0pt} \setlength{\abovedisplayshortskip}{0pt}
%\setlength{\belowdisplayskip}{0pt} \setlength{\belowdisplayshortskip}{0pt}
%\titlespacing{\subsubsection}{0pt}{1ex}{0ex}

\usepackage[bindingoffset=6mm]{geometry}

% moje libki end


%dane do złożenia strony tytułowej
%\title{Tytuł pracy inżynierskiej}
%\engtitle{English title}
\title{Filtry cząsteczkowe i ich zastosowania w~problemach wyznaczania lokalizacji}
\engtitle{Particle filters for selected problems in positioning}
\author{Wojciech Sopot}
\supervisor{dr hab. inż. Paweł Wachel, prof. ucz., W04/K8}
%\guardian{dr hab. inż. Imię Nazwisko Prof. PWr, I-6} %nie używać jeśli opiekun jest tą samą osobą co prowadzący pracę

%\date{2008} %standardowo u dołu strony tytułowej umieszczany jest bieżący rok, to polecenie pozwala wstawić dowolny rok

%poniżej jest lista kierunków i specjalności na wydziale elektroniki, należy wybrać właściwe lub dopisać jeśli nie ma odpowiednich
\field{Automatyka i Robotyka (AIR)}
%\specialisation{Robotyka (ARR)}
%\specialisation{Komputerowe sieci sterowania (ARK)}
\specialisation{Systemy informatyczne w automatyce (ASI)}
%\specialisation{Komputerowe systemy zarządzania \\procesami produkcyjnymi (ARS)}
%\field{Elektronika i telekomunikacja (EIT)}
%\specialisation{Akustyka (ETA)}
%\specialisation{Aparatura elektroniczna (EAE)}
%\specialisation{Elektroniczne i komputerowe \\systemy automatyki (ESA)}
%\specialisation{Zastosowania inżynierii komputerowej \\w technice (EZI)}
%\specialisation{Inżynieria dźwięku (EID)}
%\specialisation{Elektronika stosowana \\i optokomunikacja (TEO)}
%\specialisation{Telekomunikacyjne sieci szerokopasmowe (TSS)}
%\specialisation{Teleinformatyczne sieci mobilne (TSM)}
%\specialisation{Sygnały w telekomunikacji cyfrowej (TSC)}
%\specialisation{Teleinformatyczne systemy rozsiewcze (TSR)}
%\field{Informatyka (INF)}
%\specialisation{Systemy informatyki w medycynie \\i technice (IMT)}
%\specialisation{Inżynieria systemów informatycznych (INS)}
%\specialisation{Inżynieria internetowa (INT)}
%\specialisation{Systemy i sieci komputerowe (ISK)}
%\field{Teleinformatyka (TIN)}
%\specialisation{Teleinformatyka (TIN)}

%tutaj zaczyna się właściwa treść dokumentu
\begin{document}
\bibliographystyle{plabbrv} %tylko gdy używamy BibTeXa, ustawia polski styl bibliografii

\maketitle %polecenie generujące stronę tytułową
%\dedication{6cm}{To jest przykładowa treść opcjonalnej dedykacji, należy ją zmienić lub usunąć w całości polecenie \texttt{$\backslash$dedication}}

\tableofcontents %spis treści
	
\chapter{Wprowadzenie}
W rozdziale przedstawiono cel i zakres pracy oraz podstawowe zadania i problemy wyznaczania lokalizacji. Ponadto opisano ogólną ideę stojącą za filtrami cząsteczkowymi - filtrację Bayesowską. Praca jest inspirowana \cite{appl3}.
\section{Zadania i problemy wyznaczania lokalizacji}
Mogłoby się wydawać, iż w dzisiejszych czasach, gdy mamy możliwość korzystania z systemu GPS, nie ma potrzeby zajmować się innymi sposobami wyznaczania lokalizacji. Jednak o ile jest to prawdą w dużej skali, jak np. gdy chce się wyznaczyć adres w mieście pod którym się znajdujemy, to gdy chcemy wyznaczyć swoją lokalizację bardziej dokładnie, np. jak to robią niektóre odkurzacze mobilne, to trzeba wykorzystać do tego dane z~innych sensorów, np. lidarów. Czasami można wcale nie mieć możliwości korzystania z~systemu GPS, na przykład pod wodą. Innym problemem może być poprawa już znanego przybliżonego położenia. W pracy zostaną przeanalizowane dwa problemy związane z~wyznaczaniem lokalizacji:
\begin{itemize}
	\item Określanie położenia samolotu, na podstawie znanej mapy wysokościowej terenu, odczytów z wysokościomierza barometrycznego, oraz, ewentualnie, dodatkowych danych (np. odczytu z kompasu). 
	\item Określanie lokalizacji robota mobilnego, umieszczonego w ograniczonej przestrzeni z przeszkodami (ale na płaskiej powierzchni - mapa jest w dwóch wymiarach).
\end{itemize}
Mimo tego, iż są to jedynie dwa przypadki, w praktyce można je uogólnić na wiele sytuacji. Dla przykładu w \cite{underwater_pf} zajęto się problemem lokalizacji pod wodą, który w praktyce można rozwiązać w ten sam sposób co problem samolotu - mapa wysokości zostaje jedynie zastąpiona przez mapę głębokości.
\section{Cel pracy}
Celem pracy jest rozpoznanie możliwości zastosowania filtrów cząsteczkowych do rozwiązywania problemu lokalizacji oraz samodzielne zaimplementowanie i~przebadanie kilku, uznanych arbitralnie przez autora za najciekawsze.
\section{Zakres pracy}
W zakres pracy wchodzi przegląd literatury na temat filtrów cząsteczkowych pod kątem problemów wyznaczania lokalizacji, następnie wybranie kilku rozwiązań oraz ich przebadanie. Aby było to możliwe konieczny jest także przegląd znanych narzędzi programistycznych, które mogą być zastosowane do zaimplementowania algorytmów.

\section{Idea filtracji Bayesowskiej}
Na chwilę obecną, algorytmy oparte o filtrację Bayesowską są szeroko wykorzystywane w automatyce i robotyce. W skrócie, podejście to polega na przeprowadzaniu cykli predykcji i poprawek, na podstawie rozkładu a priori oraz zbieranych w kolejnych iteracjach pomiarów, w celu wyznaczenia rozkładu a posteriori stanu systemu. Ideę tego podejścia widać na rysunku \ref{bayes_fil_idea}. 
\begin{figure}[H]
	\begin{center}
		\includegraphics[width=10cm]{./predict_update.png}
		\caption[Idea filtracji Bayesowskiej]{Idea filtracji Bayesowskiej. Obrazek zaczerpnięto z \ref{main_idea}.}
		\label{bayes_fil_idea}
	\end{center}
\end{figure}
Na rysunku \ref{filtr_hier} przedstawiono hierarchię rozwiązać opartych o to podejście. Jak widać jest to bardzo duża rodzina rozwiązań, jednak w pracy zostaną poruszone jedynie metody oparte o filtry cząsteczkowe.
\begin{figure}[H]
	\begin{center}
		\includegraphics[width=10cm]{./nfg001.jpg}
		\caption[Hierarchia filtrów opartych o filtrację Bayesowską.]{Hierarchia filtrów opartych o filtrację Bayesowską. Obrazek zaczerpnięto z \cite{prac_gui}.} \label{filtr_hier}
	\end{center}
\end{figure}

\chapter{Filtry cząsteczkowe}
%\subsection{Pojęcia podstawowe}
W tym rozdziale został opisany podstawowy algorytm filtra cząsteczkowego, wraz z jego adaptacją do wyznaczania lokalizacji. Ponadto opisano tutaj kilka wybranych usprawnień i modyfikacji podstawowej wersji.
\section{Opis problemu}
Celem filtru cząsteczkowego jest wyznaczenie rozkładu zmiennej stanu $X_k$, na podstawie szczątkowych pomiarów $Y_k$ za pomocą filtracji Bayesowskiej. \\
\begin{equation*}
\begin{tikzcd}
	X_0 \arrow{r} \arrow{d} &X_1 \arrow{r} \arrow{d} &X_2 \arrow{r} \arrow{d} &X_3\arrow{r} \arrow{d} &...\\
	Y_0 &Y_1 &Y_2 &Y_3 &...
\end{tikzcd} 
\end{equation*}
System taki, można opisać w następujący sposób
\begin{equation*}
	\begin{aligned}
		X_k=g(X_{k-1})+W_{k-1} \\
		Y_k=h(X_k)+V_k
	\end{aligned}
\end{equation*}
gdzie $g$ i $h$ to znane funkcje, a $W_{k-1}$ i $V_k$ to szumy. Jeśli oba szumy byłyby z rozkładu normalnego, a $g$ i $h$ są liniowe, to problem można rozwiązać za pomocą filtru Kalmana. Aby zyskać na ogólności, system można opisać w następujący sposób:
\begin{equation} \label{problem_eq}
	\begin{aligned}
		X_k=g(X_{k-1}, W_{k-1}) \\
		Y_k=h(X_k, V_k)
	\end{aligned}
\end{equation}
Taki problem, gdy funkcje $g$ i $h$ mogą być nieliniowe względem $X_k$, $Y_k$ oraz szumów $W_{k-1}$ i $V_k$, zazwyczaj jest zbyt złożony aby można było zastosować podejścia oparte o filtr kalmana. W takich przypadkach jedną z możliwości jest zastosowanie filtrów cząsteczkowych. Warto zauważyć, iż funkcja $g$ zawiera informacje o sterowaniu.
\section{Matematyczny opis filtracji Bayesowskiej}
Ogólne wzory na filtrowanie Bayesowskie można znaleźć na przykład na wikipedi \cite{wiki_bayes_filter}. Celem pojedynczego kroku filtracji jest wyznaczenie rozkładu a posteriori $p(X_{k}|Y_{0...k})$ opisującego prawdopodobieństwo, że system jest w stanie $X_k$, przy założeniu historii pomiarów $Y_{0...k}$, na podstawie rozkładu a priori $p(X_{k-1}|Y_{0...k-1})$ uzyskanego w poprzednim kroku oraz nowego pomiary $Y_k$. Etap predykcji można opisać poniższym równaniem:
\begin{equation} \label{evolution}
	p(X_k|Y_{0...k-1})=\int p(X_k|X_{k-1})p(X_{k-1}|Y_{0...k-1}) dX_{k-1}
\end{equation}
gdzie $p(X_k|X_{k-1})$ opisuje ewolucję systemu między krokami $k-1$ i $k$. Etap korekcji realizuje się w oparciu o wzór Bayesa, po otrzymaniu kolejnego pomiaru $Y_k$:
\begin{equation}\label{bayes_formula}
	p(X_k|Y_{0...k})=\frac{p(Y_k|X_k)p(X_k|Y_{0...k-1})}{p(Y_k|Y_{0...k-1})}
\end{equation}
gdzie $p(Y_k|X_k)$ opisuje rozkład prawdopodobieństwa uzyskania pomiaru $Y_k$ w stanie $X_k$. Warto zauważyć, iż $p(Y_k|Y_{0...k})$ można opisać wzorem:
\begin{equation}
p(Y_k|Y_{0...k})=\int p(Y_k|X_k)p(X_k|Y_{0...k-1}) dX_k
\end{equation}
więc mianownik równania \ref{bayes_formula} jest stały względem $X_k$. Mając to na uwadze, oraz podstawiając równanie \ref{evolution} do \ref{bayes_formula} można uzyskać ogólną formułę jednego kroku filtracji:
\begin{equation}
	p(X_k|Y_{0...k})=\mu p(Y_k|X_k)\int p(X_k|X_{k-1})p(X_{t-1}|Y_{0...t-1}) dX_{t-1}
\end{equation}
gdzie $\mu$ jest stałą normalizacyjną, pozostałą po przyjęciu że $p(Y_k|Y_{0...k})$ ma stałą wartość.
\section{Reprezentacja rozkładu prawdopodobieństwa}
Aby móc w praktyce zrealizować filtr cząsteczkowy, konieczne jest znalezienie sposobu na cyfrową reprezentację ciągłego rozkładu prawdopodobieństwa związanego ze stanem systemu. Robi się to przybliżając go za pomocą skończonego zbioru cząstek, gdzie każda cząstka wygląda w następujący sposób:
\begin{equation*}
	p_{i,k}=\{x_{k,i},w_{k,i}\}
\end{equation*}
gdzie $x_{k,i}$ to stan w iteracji $k$ związany z $i$-tą cząstką, $w_{k,i}$ to jej waga. Korzystając z takiej reprezentacji, rozkład $p(X_k|Y_{0...k})$ ze wzoru \ref{evolution} można zapisać w następujący sposób:
\begin{equation}
	p(x_k|Y_{0...k})\approx \sum_{i=0}^{N} w_{k,i}	\delta(X_k-x_{k,i})
\end{equation}
gdzie $N$ oznacza liczbę cząstek, a $\delta$ to delta Dirac'a. Dzięki takiej reprezentacji prawdopodobieństwo $p(X_k|Y_{0...k-1})$ opisane we wzorze \ref{evolution} zostaje zastąpione przez wagę $w_{k,i}$ dla każdej cząstki, co sprawia, że równanie \ref{bayes_formula} upraszcza się do:
\begin{equation}\label{weight_update}
	w_{k,i} = w_{k-1,i} p(y_k|x_{k,i})
\end{equation}

% Warto zauważyć, że taka reprezentacja realizuje procedurę importance sampling

\section{Podstawowy algorytm - SIR} \label{basic_algorithm}
Sam skrót SIR oznacza stochastic importance resampling, i jest to standardowa realizacja filtra cząsteczkowego w praktyce. Opis algorytmu można znaleźć w \cite{wiki_pf}. Sam importance sampling polega na wykorzystaniu właściwości jednego rozkładu prawdopodobieństwa, do próbkowania z drugiego. W tym przypadku, próbkujemy z rozkładu $p(X_k|Y_{0...k})$ korzystając $p(Y_k|X_k)$.
Algorytm SIR Przebiega w następujący sposób:
\begin{enumerate}[label=(\alph*)]
	\item Zaczyna się od zainicjowania zbioru cząstek losowymi stanami i równymi dla wszystkich cząstek wagami. Ważne jest, aby uzyskana w ten sposób populacja była w stanie odpowiednio oddać właściwości niezdyskretyzowanego rozkładu. \label{pf_init_step}
	
	\item Symuluje się ewolucję cząstek $x_{k-1,i}$, próbkując $x_{k,i}$ z rozkładu opisującego ewolucję systemu, $p(x_k|x_{k-1})$, obecnego w równaniu \ref{evolution}, a dobranemu na podstawie funkcji $g$ z równania \ref{problem_eq}. W praktyce ważne jest, aby rozkład $p(x_k|x_{k-1})$ uwzględniał szumy obecne w systemie, co sprowadza się do tego, że do ewolucji wprowadza cię trochę losowości. \label{pf_drift_step}
	
	\item Pojawia się nowy pomiar $y_k$, na podstawie którego na którego podstawie modyfikuje się wagi poszczególnych cząstek, korzystając z równania \ref{weight_update}. \label{pf_reweight_step}

	\item Wagi są normalizowane. \label{pf_weight_normalization_step}
	\begin{equation}
		w_{k,i}=\frac{w_{k,i}}{\sum_{j=0}^{N} w_{k,j}}
	\end{equation}

	\item Wyznacza się estymowany stan, wyznaczając ważoną średnią stanów wszystkich cząstek. Zazwyczaj jest to średnia ważona cząstek: \label{pf_est_step}
	\begin{equation}
		\hat{x_k} = \sum_{i=0}^{N} w_{k,i} x_{k,i}
	\end{equation}
	\item Przeprowadza się ponowne próbkowanie, generując nową populację cząstek z poprzedniej, korzystając z wag. W praktyce zazwyczaj nie przeprowadza się próbkowania co iterację, aby przyspieszyć obliczenia. Robi się to dopiero gdy spełnione jest pewne kryterium, którym najczęściej jest spadek efektywnej cząstek, opisanej równaniem \ref{neff_form} poniżej pewnego progu.
	\begin{equation}\label{neff_form}
		N_{eff_k} = \dfrac{1}{\sum w_{k,i}^2}
	\end{equation}
	Inne metody na określenie momentu próbkowania, można znaleźć na przykład w \cite{adaptive_resampling}.\label{resampling_step}
\end{enumerate}
\section{Adaptacja do wyznaczania lokalizacji}
Rozwiązanie problemu wyznaczania lokalizacji za pomocą filtra cząsteczkowego często jest nazywane lokalizacją Monte Carlo. Celem jest wtedy wyznaczenie pozycji i orientacji pewnego obiektu, na przykład robota. W takim przypadku rozkład $p(x_k|x_{k-1})$ opisuje model ruchu obiektu, natomiast pomiary $y_k$ są zebranymi danymi sensorycznymi. Prawdopodobieństwo $p(y_k|x_{k,i})$ wyznacza się na podstawie znanej mapy otoczenia.

\section{Przykład}\label{simple_example_chap}
Najprostszym przykładem problemu lokalizacji na którym można zademonstrować działanie filtra, jest ustalanie pozycji punktowego robota, przemieszczającego się ze stałą prędkością w kierunku ściany, mierząc swoją odległość od niej. Mapa w tym przypadku jest jednowymiarowa, a stan robota można opisać jedną liczbą zmiennoprzecinkową, mówiącą o jego pozycji. Pomiar $y_k$ jest odległością od robota do ściany na końcu drogi, natomiast jako rozkład $p(y_k|x_{k,i})$ przyjęto rozkład Gaussa. Na rysunku \ref{simple_example} przedstawiono efekt realizacji filtra cząsteczkowego. W tym przykładzie próbkowanie przeprowadzano na koniec każdej iteracji.

\begin{figure}[H]
	\begin{center}
		\includegraphics[width=10cm]{./simple_example.png}
		\caption{Przykładowa realizacja prostego filtra. Czerwona strzałka wskazuje kierunek ruchu robota, czerwone punkty pokazują estymowaną lokalizację robota, natomiast niebieskie punkty prawdziwe położenie. W kolejnych rzędach zestawiono histogramy położenia cząstek w poszczególnych iteracjach. Populacja składała się z 10 osobników.}\label{simple_example}
	\end{center}
\end{figure}

\section{Adaptacyjna liczba cząstek} \label{adaptive_chapter}
Oczywistym jest, że największy wpływ na złożoność obliczeniową filtra cząsteczkowego ma ilość cząstek. W \cite{adaptive} opisano algorytm który służy do dynamicznego zmieniania liczby cząstek. Polega on na zmodyfikowaniu kroku \ref{resampling_step} podstawowego algorytmu, gdzie tuż przed próbkowaniem wyznacza się nową liczbę cząstek. Wykonuje się to, kolejno usuwając cząstki, i sprawdzając o ile pogorszyła się jakość estymacji. Miarę tego pogorszenia mierzy się za pomocą funkcji opisanej poniższym równaniem
\begin{equation}\label{ceana_pogorszenia}
	\xi_t = |y_t-h(\hat{x_k})|
\end{equation}
i wyznacza się się je za każdym razem gdy usunie się cząstkę.

\begin{algorithm}[H]
	\SetAlgoLined
	\DontPrintSemicolon
	\caption{Algorytm dynamicznego doboru liczby cząstek.} \label{adaptive_N}
	Inicjalizacja $S_k=\{1,...,N_k\}$ i $K_k=\emptyset$\;
	Wyznaczenie $\xi(N_k)$\;
%	\begin{equation*}\text{Testowe usuwanie cząstek}\end{equation*}\;
	-------------------Testowe usuwanie cząstek\;
	\For{$d=0$ \KwTo $dim(x_{k,i})$}{
		Sortowanie indeksów cząstek S względem wymiaru $d$\;
		\For{$i \in S$}{
			\If{$||x_{k,i}^d-x_{k,i+1}^d||<\gamma_d$}{
				\begin{equation*}
					K_t = \begin{cases}
						K_t \cup \{i+1\} \text{ jeśli } w_{k,i}>w_{k,i+1}\\
						K_t \cup \{i\} \text{ w przeciwnym wypadku}
					\end{cases}\;
				\end{equation*}
				$S_t=S_t\slash K_t, n_s=dim\{S_t\}$\;
				$\hat{x_k} = \sum_{j \in S} w_{k,j} x_{k,j}$\;
				Wyznaczenie $\xi(n_s)$ ze wzoru \ref{ceana_pogorszenia} \;
			}
		}
	}

	-------------------Wyznaczanie nowego $N_k$\;
	\eIf{$\xi_k(n)>\alpha, \forall n \in [n_s,N_k]$}{
		$N_{k+1} \sim U(N_t,N_{max})$\;
	}{
		$N_{k+1}=argmin_n\{\xi(n)\}$ taki że $\xi(n)<\alpha$\;
	}
	\If{$N_{k+1}<N_{min}$}{
		$N_{k+1}=N_{min}$\;
	}
	
\end{algorithm}
W algorytmie \ref{adaptive_N} wydzielono dwie sekcje, testowe usuwanie cząstek oraz wyznaczanie nowego $N_k$. W pierwszym etapie wyznacza się wartości $\xi(n_s)$ dal różnych rozmiarów populacji, kolejno ujmując cząstki które leżą zbyt blisko siebie. Minimalna odległość na danym wymiarze $d$, po przekroczeniu której usuwa się jedną z cząstek, jest określana przez parametr $\gamma_d$. W drugim etapie wyznacza się nowy rozmiar populacji $N_{k+1}$ w zależności od $\xi_k$. Dopuszczalny błąd wprowadzany przez zmniejszenie populacji jest określany przez $\alpha$.

% Może się wydawać, iż zwiększy to złożoność obliczeniową, ponieważ trzeba o wiele więcej czasu poświęcić na analizę poszczególnych cząstek, jednak dzięki temu możemy zredukować samą ich liczbę, co sprawia, że koniec końców złożoność obliczeniowa nie zmienia się znacząco, a jakość wyników się poprawia
\section{Box Particle Filter} \label{bpf_chapter}
W tym podejściu zaproponowanym w \cite{bpf_base} i udoskonalonym w \cite{brbpf} zmieniono sposób reprezentacji cząstek. Zamiast punktów w przestrzeni stanów zastosowano interwały, w celu poprawy jakości estymacji w sytuacjach, gdy pomiary są zbyt szybkozmienne. Na przykład, dla problemu opisanego w podrozdziale \ref{simple_example_chap}, zamiast liczby rzeczywistej opisującej położenie robota, mielibyśmy odcinek na którym może się znajdować. Ponieważ zmianie ulega reprezentacja cząstek, wszystkie kroki algorytmu opisanego w rozdziale \ref{basic_algorithm} muszą zostać odpowiednio dostosowane.
\begin{itemize}
	\item W kroku \ref{pf_init_step}, cząsteczki interwałowe są inicjalizowane w taki sposób, aby pokrywać całą dopuszczalną przestrzeń stanów i nie nachodzić na siebie nawzajem.
	\item Ewolucja systemu opisana w kroku \ref{pf_drift_step}, zostaje zastąpione przez przekształcenie interwałów. Wykorzystuje się do tego funkcję interwałową $[g]$ konstruowaną na podstawie funkcji $g$ z równania \ref{problem_eq}. Ważne jest aby w przeciwieństwie do operacji z kroku \ref{pf_drift_step}, $[g]$ dawała w pełni deterministyczne wyniki, uwzględniające szumy obecne w systemie. Na przykład w problemie opisanym w rozdziale \ref{simple_example_chap}, jeśli cząstka byłaby interwałem [2,3] i miałaby być przesunięta o $1\pm 0.1$ to korzystając z reguły $3\sigma$ \cite{3_sigma_rule} można skonstruować nowy interwał po przesunięciu [2.9,4.1]. Opisuje się to następującym wzorem:
	\begin{equation}
		[x_{k+1,i}] = [g]([x_{k,i}])
	\end{equation}
	
	\item Zamiast pomiaru z kroku \ref{pf_reweight_step} wykorzystuje się interwał pomiarowy, uwzględniający jego niepewność (interwał ten można skonstruować według reguły $3\sigma$), oraz konstruując funkcję $[h]$, na podstawie funkcji $h$ z równania \ref{problem_eq}, generującą interwał pomiaru $[y]$ ze stanu danej cząstki. Nowe wagi ustala się na podstawie poniższych wzorów.
	\begin{equation}
		\begin{aligned}
			[z_{k,i}] = [h]([x_{k-1,i}])\\
			[r_{k,i}] = [z_{k,i}] \cap [y_{k}]\\
			w_{k,i} = \frac{|[r_{k,i}]|}{|[z_{k,i}]|} w_{k-1,i}
		\end{aligned}
	\end{equation}
	gdzie $[z_{k,i}]$ jest przewidywanym interwałem pomiaru dla cząstki $i$ w kroku $k$, $[y_{k}]$ jest interwałem faktycznie zmierzonego pomiaru, a $|\bullet|$ jest objętością interwału (dla przykładu z rozdziału \ref{simple_example_chap} byłaby to długość odcinka).
	\item Wagi są normalizowane tak samo jak w punkcie \ref{pf_weight_normalization_step} podstawowego algorytmu.
	
	\item Przed wyliczeniem estymowanego stanu można przeprowadzić zawężenie interwałów. Polega ono na pomniejszeniu $[x_{k,i}]$ w oparciu o interwał $[r_{k,i}]$. W przykładzie z rozdziału \ref{simple_example_chap}, mogło by to wyglądać w następujący sposób: jeśli uzyskano pomiar [7,9] który można uzyskać tylko w interwale [4,6], to cząstka z interwałem [3,5] zostanie zawężona do [4,5].
	
	\item Estymacja stanu odbywa się z wykorzystaniem średniej ważonej centrów interwałów (dla przykładu z rozdziału \ref{simple_example_chap} byłby to środek odcinka).
	\begin{equation}
		\hat{x_k} = \sum_{i=0}^{N} w_{k,i} C_{k,i}
	\end{equation}
	gdzie $C_{k,i}$ jest centrum $i$-tej cząstki w $k$-tej iteracji.
	\item Ponowne próbkowanie przebiega zupełnie inaczej niż w kroku \ref{resampling_step} podstawowego. Sam moment ponownego próbkowania można nadal dobierać według wzoru \ref{neff_form}. Próbkowanie zostaje zastąpione podziałem wybranych cząstek. Dla przykładu, gdy dana cząstka po próbkowaniu powinna być wzięta pięć razy, zostaje ona pięć razy podzielona wzdłuż najdłuższego wymiaru (może też być dzielona wzdłuż losowo wybranego interwału).
	\item Jednym z najważniejszych usprawnień zaproponowanych w \cite{brbpf} jest reinicjalizacja cząstek, w przypadku gdy suma wag spadnie do zera. Jest to konieczne, ponieważ w takim przypadku niemożliwe staje się próbkowanie.
\end{itemize}

\section{Wprowadzenie operatora mutacji} \label{evol_chap}
W \cite{pfgen} zaproponowano, aby wzbogacić proces próbkowania z kroku \ref{resampling_step} podstawowego algorytmu o operatory krzyżowania i mutacji. W kontekście wyznaczania lokalizacji, krzyżowanie można zaimplementować jako liniową interpolację dwóch losowo wybranych cząstek, natomiast mutację jako niewielkie zaszumienie cząstek. Pozwala to na poprawę działania filtra przy bardzo powolnej ewolucji, bądź nawet jej braku.

\chapter{Wybrane narzędzia programistyczne}
\chapter{Badania} \label{przeg}
W tym rozdziale opisano przeprowadzone na kodzie badania. Zaczęto od opisania konkretnych problemów którymi się zajmowano, a następnie przeprowadzono badania mające potwierdzić czy algorytmy działają poprawnie. Na koniec przebadano zachowanie zaimplementowane filtry pod różnymi kątami.
\section{Opis poszczególnych problemów}
W badaniach zajęto się dwoma problemami wyznaczania lokalizacji. Pierwszy polega na ustalenie pozycji robota na podstawie pomiaru odległości od ściany, drugi na ustalenie pozycji samolotu na podstawie pomiaru wysokości. W oby przypadkach znano mapę środowiska w którym się znajdowały.
\subsection{Robot w pomieszczeniu} \label{robot_w_pomieszczeniu_desc}
Stan robota w pomieszczeniu opisano czterema liczbami:
\begin{equation*}
	x = \{p_x,p_y,\theta,v\}
\end{equation*}
gdzie $p_x$ i $p_y$ to pozycja robota, $\theta$ jest jego orientacją, natomiast $v$ prędkością. Pomiar odległości od ściany był zawsze wykonywany w kierunku $\theta$, i był miał Gaussowski rozkład. Mapa jest kwadratowym pokojem o wymiarach $1000$ na $1000$ jednostek, wypełnionym kołami o różnych średnicach, oraz ograniczany prostymi. Na rysunku \ref{przykladowa_mapa_pokoju} przedstawiono przykładową mapę.\\
O ile nie będzie napisane inaczej, robot zaczynał na środku mapy, miał prędkość 10 jednostek na iterację, i skręcał w lewo o $0.1rad$ na krok.
\begin{figure}
	\begin{center}
		\includegraphics[width=10cm]{./przykladowa_mapa_pokoju.png}
		\caption{Przykładowa mapa pokoju}
		\label{przykladowa_mapa_pokoju}
	\end{center}
\end{figure}
\subsection{Samolot w locie}
Stan samolotu opisano tak samo jak stan robota w rozdziale \ref{robot_w_pomieszczeniu_desc}:
\begin{equation*}
	x = \{p_x,p_y,\theta,v\}
\end{equation*}
Mapa jest natomiast mapą wysokościową, w tym przypadku zdecydowano się na dwa warianty, pierwszy jest mapą fragmentu Wrocławia (przykład widoczny na rysunku \ref{przykladowa_mapa_wroclawia}), natomiast drugą wygenerowano jako szum, którego zmienność można kontrolować (przykład widoczny na rysunku \ref{przykladowa_mapa_szumu}).\\
O ile nie będzie napisane inaczej, samolot w lewym dolnym rogu mapy, miał prędkość 10 jednostek na iterację, i nie zmieniał orientacji $\theta=\frac{\pi}{4}$.
\begin{figure}
	\begin{center}
		\includegraphics[width=10cm]{./przykladowa_mapa_wroclawia.png}
		\caption{Przykładowa mapa wysokościowa fragmentu Wrocławia}
		\label{przykladowa_mapa_wroclawia}
	\end{center}
\end{figure}
\begin{figure}
\begin{center}
	\includegraphics[width=10cm]{./przykladowa_mapa_szumu.png}
	\caption{Przykładowa mapa wysokościowa uzyskana z szumu}
	\label{przykladowa_mapa_szumu}
\end{center}
\end{figure}

\section{Badania poprawności działania}
W tym rozdziale zajęto się badaniami mającymi potwierdzić poprawne działanie zaimplementowanego rozwiązania. Na początku sprawdzono, jak zmiana generatora liczb losowych wpłynęła na wyniki, następnie sprawdzono, jak poradzi sobie algorytm przy braku punktów odniesienia, potem co się dzieje przy braku ewolucji systemu ($v=0$), a na koniec jak filtr radzi sobie z całkowicie losowymi pomiarami. Nie skupiano się no konkretnych wartościach liczbowych, jedynie wizualnie oceniano wyniki.

\subsection{Wpływ generatora}
Przebadano trzy generatory wbudowane w język C++: domyślny linear congruential generator(LCG) \cite{lcg_wiki}, Mersenne Twister \cite{mersenne_wiki} oraz wbudowany niedeterministyczny generator. Populacja cząstek wynosiła $N=1000$. Wyniki rysowano po 1, 11 i 21 iteracjach filtra(nr iteracji oznaczano jako $k$). Wyniki przedstawiono na rysunkach \ref{lcg_example}, \ref{mersenne_example}, \ref{device_example}. Jak widać dla wszystkich generatorów wyniki są niemal takie same, i zbiegają do faktycznego położenia robota. Ponieważ dla każdego generatora uzyskano poprawne wyniki, w dalszych badaniach korzystano z generatora LCG ponieważ jest najprostszy, i, co za tym idzie, najszybszy.

\begin{figure}
	\begin{center}
		\includegraphics[width=15cm]{./lcg_example.png}
		\caption{Przykładowe wyniki dla generatora LCG}
		\label{lcg_example}
	\end{center}
\end{figure}

\begin{figure}
	\begin{center}
		\includegraphics[width=15cm]{./mersenne_example.png}
		\caption{Przykładowe wyniki dla generatora Mersenne Twister}
		\label{mersenne_example}
	\end{center}
\end{figure}

\begin{figure}
\begin{center}
	\includegraphics[width=15cm]{./device_example.png}
	\caption{Przykładowe wyniki dla niedeterministycznego generatora}
	\label{device_example}
\end{center}
\end{figure}


\subsection{Niejednoznaczna mapa}
W tym przypadku zbadano jak zachowa się filtr przy braku punktów odniesienia. W kwadratowym pokoju, powinno to spowodować pojawienie się czterech równie prawdopodobnych pozycji. Jak widać przewidywania potwierdziły się na rysunku \ref{no_pivot}. Na rysunku \ref{one_pivot} przedstawiono sytuację, gdy mapa dopuszcza dwie możliwe pozycja. Badania przeprowadzono przy populacji $N=10000$ cząstek.

\begin{figure}
	\begin{center}
		\includegraphics[width=15cm]{./no_pivot.png}
		\caption{Przykładowe wyniki przy braku punktu odniesienia}
		\label{no_pivot}
	\end{center}
\end{figure}

\begin{figure}
	\begin{center}
		\includegraphics[width=15cm]{./one_pivot.png}
		\caption{Przykładowe wyniki przy dwóch możliwych położeniach}
		\label{one_pivot}
	\end{center}
\end{figure}

\subsection{Brak ewolucji systemu}
Tutaj Zbadano, co się dzieje, gdy system nie ewoluuje. Spodziewano się, pojawią się izohipsy, przynajmniej na początku, nim populacja nie stanie się zbyt zdegenerowana. Badania przeprowadzono dla populacji o rozmiarze $N=100000$. Jak widać na rysunkach \ref{stationary} i \ref{stationary_plane} przewidywania się spełniły, dodatkowo dla rysunku \ref{stationary_plane} dobrze widać pojawianie się degeneracji.

\begin{figure}
	\begin{center}
		\includegraphics[width=15cm]{./stationary.png}
		\caption{Przykładowe wyniki przy braku ewolucji systemu, dla robota w pokoju}
		\label{stationary}
	\end{center}
\end{figure}

\begin{figure}
\begin{center}
	\includegraphics[width=15cm]{./stationary_plane.png}
	\caption{Przykładowe wyniki przy braku ewolucji systemu, dla samolotu}
	\label{stationary_plane}
\end{center}
\end{figure}

\subsection{Pomiary bez związku}
powinien pozostać jednostajny szum

\section{Wpływ sposobu estymowania położenia}
\section{Wpływ funkcji określającej błąd pomiaru}
\section{Wpływ liczby cząstek}
\section{Wpływ metody próbkowania}
\section{Skuteczność w poprawianiu błędnie określonego położenia}
\section{Wpływ szumu}
\section{Wpływ różnorodności terenu}
\section{Rozkład p przy dryfie - chodzi o genetyczne}


\chapter{Podsumowanie}
\section{Wnioski}
\begin{itemize}
	\item 
\end{itemize}

\section{Kierunki przyszłych badań}
\begin{itemize}
	\item Zrezygnowanie z implementacji C++ na rzecz całkowitego przejścia na Python, z wykorzystaniem na przykład pakietu Numba \cite{numba}. Poza przyspieszeniem porównywalnym z C++, pozwoliło by na przykład na poszerzanie kodu o obliczenia na karcie graficznej. Dzięki temu, można by spróbować rozwiązać problem robota w pokoju, wykorzystując Box Particle Filter.
	\item zamiast avg, np klasteryzacja?
\end{itemize}




%%%%%%%%%%%%%%%%%%%%%%%%%%%%%%%%%%%%%%%%%%%%%%%%%%%%%%%%%%%%%%
%%%%%%%%%%%%%%%%%%%%%%%%%%%%%%%%%%%%%%%%%%%%%%%%%%%%%%%%%%%%%%
%%%%%%%%%%%%%%%%%%%%%%%%%%%%%%%%%%%%%%%%%%%%%%%%%%%%%%%%%%%%%%

%~\cite{WinNT}

\appendix
\chapter{Zawartość załączonej płyty CD}
\begin{itemize}
    \item Praca magisterska - dokument w formacie .pdf
    \item Przygotowane oprogramowanie - pliki w formacie .cpp oraz setup.py
    \item Przygotowane demonstracje - pliki w formacie .py poza setup.py
    \item requirements.txt - opis środowiska Pythonowego
\end{itemize}

\addcontentsline{toc}{chapter}{\bibname} %utworzenie w spisie treści pozycji Literatura
\bibliography{biblio}{} % wstawia bibliografię korzystając z pliku bibliografia.bib - dotyczy BibTeXa, jeżeli nie korzystamy z BibTeXa należy użyć otoczenia thebibliography

%opcjonalnie może się tu pojawić spis rysunków i tabel
 \listoffigures
 %\listoftables

\end{document}
