\chapter{Przygotowane oprogramowanie}
W tym rozdziale opisano jakie narzędzia programistyczne wybrano i dlaczego akurat te, a następnie opisano przygotowane oprogramowanie.
\section{Wybrane narzędzia programistyczne}
Najważniejszymi kryteriami branymi pod uwagę przy wyborze języka programowania była wydajność, wcześniejsza jego znajomość. Jeśli chodzi o IDE(integrated development environment) wystarczyło jeśli dało się w nim pracować z wybranymi językami. Kierując się wydajnością, jako język do implementacji algorytmów wybrano C++, jednak ze względu na brak prostego sposobu graficznej prezentacji wyników, pomocniczo skorzystano z języka Python i jego bibliotek numpy\cite{numpy} i matplotlib\cite{matplotlib}. Aby połączyć oba języki skorzystano z biblioteki pybind11\cite{pybind11} która umożliwia proste tworzenie bibliotek dla Pythona napisanych w C++. W konsekwencji pociągnęło to za sobą wybór IDE - edytora tekstu Sublime Text\cite{sublime}, ponieważ częsta instalacja biblioteki wymuszała ciągłe korzystanie z wiersza poleceń. Aby ułatwić zarządzanie środowiskami Pythona, oraz samą z nim pracę, wykorzystano pakiet Anaconda\cite{anaconda}. W celu zaimplementowania algorytmu Box Particle Filter skorzystano z biblioteki interval będącej częścią zbioru bibliotek Boost\cite{boost_int}.
\section{Opis przygotowanej biblioteki}
Cała biblioteka została zorganizowana jako zbiór funkcji umożliwiających zarządzanie tablicami stanów, gdzie stany były prostymi strukturami danych, oraz kilkoma klasami: klasą implementującą Box Particle Filter, oraz klasami realizującymi mapy potrzebne do badań. Poniżej ogólnie przedstawiono ich opis (dokładniejsze informacje można uzyskać analizując kod), zaś sama biblioteka nazywa się PFlib:
\begin{itemize}
	\item robot\_2d - struktura służąca za kontener na stan robota opisany w rozdziale \ref{robot_w_pomieszczeniu_desc}.
	\item roulette\_wheel\_resample - funkcja realizująca próbkowanie ruletkowe opisane w \cite{rou_wiki} w oparciu o wagi.
	\item sus\_resample - funkcja realizująca próbkowanie o niskiej wariancji opisane w \cite{sus_wiki} w oparciu o wagi.
	\item as\_array - funkcja konwertująca tablicę stanów na macierz.
	\item get\_uniform\_weights - funkcja inicjalizująca tablicę równych sobie wag.
	\item get\_est - funkcja wyznaczająca średnią ważoną stanów dla danej tablicy stanów i wag.
	\item update\_weights - funkcja przeliczająca wagi z wykorzystaniem nowego pomiaru, korzystając ze wzoru \ref{weight_update}.
	\item drift\_state - funkcja przeprowadzająca ewolucję pojedynczego stanu.
	\item drift\_pop - funkcja przeprowadzająca ewolucję tablicy stanów.
	\item get\_random\_pop - funkcja inicjalizująca losową populację.
	\item get\_linear\_pop - funkcja inicjalizująca populację równo rozmieszczoną wzdłuż jednaj osi.
	\item get\_new\_N - funkcja wyznaczająca nowy rozmiar populacji korzystając z algorytmu \ref{adaptive_N}.
	\item regularize - funkcja implementująca podejście opisane w rozdziale \ref{evol_chap}.
	\item BoxParticleFilter - klasa implementując podejście opisane w rozdziale \ref{bpf_chapter}, posiadająca następujące metody:
	\begin{itemize}
		\item init\_pop - metoda inicjalizująca populację interwałowych cząstek.
		\item reinit\_pop - metoda reinicjalizująca populację.
		\item update\_weights - metoda przeliczająca wagi z wykorzystaniem nowego pomiaru, korzystając ze wzoru \ref{weight_update}.
		\item drift - metoda przeprwadzająca ewolucję interwałowych cząstek.
		\item get\_est - metoda zwracająca estymowany stan, według wzoru \ref{bpf_est}
		\item resample - metoda przeprowadzająca próbkowanie o niskiej wariancji \cite{sus_wiki}.
		\item get\_coeff - metoda zwracająca współczynnik wyznaczony ze wzoru \ref{theta_coeff}
		\item get\_pop - funkcja zwracająca macierz zawierającą reprezentacje interwałowych cząstek.
	\end{itemize}
	\item PrimitiveMap - klasa realizująca mapę potrzebną do problemu opisanego w rozdziale \ref{robot_w_pomieszczeniu_desc}, posiadającą następujące metody:
	\begin{itemize}
		\item get\_grid - metoda zwracająca macierz będącą zero-jedynkową reprezentacją mapy (1 - komórka zajęta, 0 - komórka wolna).
		\item add\_line - metoda pozwalająca ograniczyć mapę prostą.
		\item add\_circle - metoda pozwalająca dodać przeszkodę w kształcie koła.
		\item get\_meas - metoda pozwalająca uzyskać pomiar na zadanych współrzędnych w danej pozycji.
	\end{itemize}

	\item HeightMap - klasa realizująca mapę potrzebną do problemu opisanego w rozdziale \ref{samolot_w_locie_chap}, posiadającą następujące metody:
	\begin{itemize}
		\item get\_grid - metoda zwracająca macierz będącą mapą wysokościową.
		\item get\_meas - metoda pozwalająca uzyskać pomiar na zadanych współrzędnych.
	\end{itemize}
	
\end{itemize}
